\documentclass[12pt,a4paper]{report}
\usepackage[utf8]{inputenc}
\usepackage[english]{babel}
\usepackage[T1]{fontenc}
\usepackage{amsmath}
\usepackage{amsfonts}
\usepackage{amssymb}
\usepackage{lmodern}
\usepackage[none]{hyphenat}
\usepackage{authoraftertitle}
\usepackage[final]{graphicx}
\DeclareGraphicsExtensions{.pdf,.png,.jpg}
\usepackage{color}
\usepackage{hyperref}
\usepackage[tikz]{bclogo}

\hypersetup{
	colorlinks=true,
	linkcolor=black,
	urlcolor=red,
	citecolor=black,
	linktoc=all
}

\usepackage{listings}
\definecolor{forestgreen}{RGB}{34,139,34}
\definecolor{orangered}{RGB}{239,134,64}
\definecolor{darkblue}{rgb}{0.0,0.0,0.6}
\definecolor{gray}{rgb}{0.4,0.4,0.4}

\lstdefinestyle{XML} {
	language=XML,
	extendedchars=true, 
	breaklines=true,
	breakatwhitespace=true,
	emph={},
	emphstyle=\color{red},
	basicstyle=\ttfamily,
	columns=fullflexible,
	commentstyle=\color{gray}\upshape,
	morestring=[b]",
	morecomment=[s]{<?}{?>},
	morecomment=[s][\color{forestgreen}]{<!--}{-->},
	keywordstyle=\color{orangered},
	stringstyle=\ttfamily\color{black}\normalfont,
	tagstyle=\color{darkblue}\bf,
	morekeywords={attribute,xmlns,version,type,release},
	otherkeywords={attribute=, xmlns=},
	basicstyle=\small
}


%Header & Footer
%\usepackage{lastpage}
%\rfoot{\thepage\ of \pageref{LastPage}}



\date{29.09.2014}
\author{Ari Ayvazyan}
\title{JEE Security Structure Part 1}

\begin{document}

\maketitle
\tableofcontents


\chapter{\MyTitle}
%\begin{flushright}
%        \MyAuthor, \MyDate
%\end{flushright}


\section{Introduction to Security Architecture}
Most web applications have a few things in common:\\
They need to figure out who is the user that is using the application and what is he allowed to do and see.\\\\
A typical application has more than one security layer, it may be protected by only being available from a specified network or VPN. In addition there usually is some kind of identity determination followed by a SQL user with permission to query only the required functions and data sets.\\
On top of this, there should be output escaping to ensure that a attacker, who is able to manipulate the output for users, is limited in the harm he is able to cause.
\begin{figure}[h]
\centering
\includegraphics[width=1\linewidth]{res/SecurityLayers}
\caption{Security Layers in a common JEE application}
\label{fig:SecurityLayers}

\end{figure}
\\According to Oracle\cite{oracleDoc}, there are two ways to implement such access control functionality with Java EE:
\begin{enumerate}
	\item Programmatic
	\item Declarative (this includes Annotations and XML-Files)
\end{enumerate}
While the programmatic implementation offers a wider range of customization, the declarative provides a well structured and easy to use approach.\\


\section{Authentication}
Authentication describes the identification process. This is mostly done by asking for a user-name \& password or sending a Token/Hash.

\section{Authorization}
Authorization is what happens after you are authenticated.\\
It deals with the question of what a authenticated person is allowed to do.\\
Authorization may be applied to URLs or resources like Beans and Servlets.\\

\begin{figure}[h]
	\centering
	\includegraphics[width=0.5\linewidth]{res/AuthenAuthor}
	\caption{Authentication \& Authorization}
	\label{fig:AuthenAuthor}
\end{figure}

\section{Deployment Descriptors}
A Deployment Descriptor describes how a Java EE application should be deployed.\\
They contain information about security constraints, accessibility and resource references.\\\\
Deployment Descriptors are XML-Files that are by default located in the /WEB-INF/ directory.\\
The following deployment descriptors may be found here:
\begin{itemize}
	\item web.xml
	\item <vendor-specific>.xml (E.g. when using Glassfish: glassfish-web.xml)
\end{itemize}

\subsection{web.xml}
The Web.xml file stores apart from usual deployment information like servlet mappings also security related information about:\\
\begin{itemize}
	\item Protected Resources
	\item Security Roles
	\item Authentication methods
\end{itemize}
The following XML snippets are located within the <web-app></web-app> tag.
\newpage
\subsubsection*{Protected Resources}
It is possible to limit access to resources by defining a security constraint on the URL or by securing the resource itself.\\
E.g. to protect the /primes/ URL with all its subdirectories we would have to use the following code:\\
\begin{bclogo}[couleur=yellow!15,arrondi=0.1,logo=\bccrayon, ombre = true]{Securing a URL}
\begin{lstlisting}[style=XML]
<security-constraint>
	<web-resource-collection>
		<web-resource-name>primes
		</web-resource-name>
<!--Include /primes/ including all following subfolders-->
		<url-pattern>/primes/*</url-pattern>
<!-- This would result in a security leak because 
	there are more http-methods than GET and POST-->
<!-- by defining no http-method at all, 
	everything will be blocked-->
		<http-method>GET</http-method>
		<http-method>POST</http-method>
	</web-resource-collection>
	
	<auth-constraint>
		<role-name>view_role</role-name>
	</auth-constraint>
</security-constraint>
\end{lstlisting}
\end{bclogo}
In result, only a user with the role view\_role is allowed to access the defined resources.
\newpage
\subsubsection*{Security Roles}
A Security Role was used in the last subsection "Protected Resources", we declared that only users with a view\_role are allowed to view the restricted URL.\\
A Security Role is a abstract layer in front of the container, it defines a identifier which we can use for constraints. This identifier is then used by the container to specify its meaning by telling who is part of this Security Role.\\
\begin{bclogo}[couleur=yellow!15,arrondi=0.1,logo=\bccrayon, ombre = true]{Defining a Security Role}
\begin{lstlisting}[style=XML]
    <security-role>
	    <description>This role has view access</description>
	    <role-name>view_role</role-name>
    </security-role>
\end{lstlisting}
\end{bclogo}
\subsubsection*{Authentication Methods}
There are several kinds of Authentication Methods with different security behaviors.\\
BASIC Authentication opens a login prompt when a user tries to access the secured URL, it is simple to implement but insecure. BASIC Authentication sends the user's credentials unencrypted to the server.\\
\begin{bclogo}[couleur=yellow!15,arrondi=0.1,logo=\bccrayon, ombre = true]{Defining BASIC Authentication}
\begin{lstlisting}[style=XML]
    <login-config>
	    <auth-method>BASIC</auth-method>
	    <realm-name>Java EE Login</realm-name>
    </login-config>
\end{lstlisting}
\end{bclogo}
\newpage
\subsection{(vendor-specific).xml}
Most containers use in addition to the web.xml a vendor specific XML file that is usually located in the same directory with web.xml.\\
E.g. Glassfish calls this file "glassfish-web.xml" while Tomcat has named it "context.xml".\\\\
The following settings can be configured there:\\
\begin{itemize}
	\item User – Role mapping
	\item Group – Role mapping
	\item Other container specific configuration.
\end{itemize}


\section{Principals}
A Principal is a identity that can be authenticated.\\
E.g. a Unique user name\\

\section{Credential}
A Credential is defined as information that is used to authenticate a Principal.\\
E.g. a Password\\

\section{Groups}
Groups and Principals can be mapped to Roles.\\
Groups are defined in the vendor-specific.xml\\

\section{Roles}
Permissions are granted to Roles.\\
Roles are defined in the web.xml file\\

\section{Realms}
aka Security policy domain\\
Provides information about principals, their Groups and their credentials\\
May be a Database, File structure, connection…\\\\
In other words:\\
It contains user information\\
E.g. Username, Password \& Permissions\\

\newpage
\section{Implementation sample}
\href{https://github.com/aayvazyan-tgm/JavaEESecurityExample}{https://github.com/aayvazyan-tgm/JavaEESecurityExample}\\
\begin{figure}[h!]
\centering
\includegraphics[width=1\linewidth]{res/Unauthorized}
\caption{The user tries to access a resource without authentication}
\label{fig:Unauthorized}
\end{figure}

\begin{figure}[h!]
	\centering
	\includegraphics[width=1\linewidth]{res/Authorized}
	\caption{The user sends authentication data with his request}
	\label{fig:Authorized}
\end{figure}

\section{Frameworks}
\subsection{Shiro}
Offers: Authentication, Authorization, Cryptography\\
Simple to use\\\\
Advantages/Disadvantages\\
Implementation Sample

\subsection{Spring}
Offers: Authentication, Authorization, Cryptography\\
Very structured\\\\
Advantages/Disadvantages\\

\subsection{JAAS - Java Authentication and Authorization Service}
Offers: Authentication, Authorization, Cryptography\\
Included in Java SE since Java 1.4 (javax.security.auth)\\\\
Advantages/Disadvantages\\

\section{Output escaping}
Escape user input to prevent injections.\\
\\
Escape the output to add a extra layer of security.\\
Use a Framework to do so!\\

\section{Whats to come in Part 2 (Adrian)}
\begin{itemize}
	\item Working with Digital Certificates\\
	\item Securing Application Clients\\
	\item Security with Enterprise Beans\\
	\item Further Framework Information\\
\end{itemize}


\newpage
\begin{thebibliography}{99}
\bibitem{JavaOne}JavaOne 2014: The Anatomy of a Secure Web Application Using Java, \\
Shawn McKinney \& John Field, September 29, 2014\\
San Francisco
\bibitem{jSecVvulnerabilities}Java Security: Sicherheitslücken identifizieren und vermeiden, \\
Marc Schönefeld, 1. edition 2011\\
Publisher: Hüthig Jehle Rehm GmbH, Heidelberg. \\
ISBN/ISSN 978-3-8266-9105-8
\bibitem{EJSec}Enterprise Java Security: Building Secure J2EE Applications,\\
Marco Pistoia, Nataraj Nagaratnam, Larry Koved, Anthony Nadalin,\\
1. edition 2004 \\
Publisher: Addison-Wesley Professional.\\
ISBN/ISSN: ISBN 0-321-11889-8
\bibitem{oracleDoc}Official JavaEE Documentation, Oracle,\\
29.09.2014
http://docs.oracle.com/javaee/7/tutorial/partsecurity.htm\#GIJRP
Java EE 6,\\
Dirk Weil, 1. edition 2012\\
Publisher: entwickler.press\\
ISBN 978-3-86802-077-9
\bibitem{jeecookbook}Java EE 6 Cookbook for Securing, Tuning, and Extending Enterprise Applications,\\
Mick Knutson,\\
1. edition June 2012 \\
Publisher: Addison-Wesley Professional.\\
ISBN/ISSN: ISBN 9781849683166
\end{thebibliography}
\end{document}

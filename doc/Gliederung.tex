\documentclass[12pt,a4paper,draft]{report}
\usepackage[utf8]{inputenc}
\usepackage[english]{babel}
\usepackage[T1]{fontenc}
\usepackage{amsmath}
\usepackage{amsfonts}
\usepackage{amssymb}
\usepackage{lmodern}
\usepackage{authoraftertitle}
\usepackage[final]{graphicx}
\DeclareGraphicsExtensions{.pdf,.png,.jpg}
\usepackage{color}
\usepackage{hyperref}

%Header & Footer
%\usepackage{lastpage}
%\rfoot{\thepage\ of \pageref{LastPage}}



\date{29.09.2014}
\author{Ari Ayvazyan}
\title{JEE Security Structure Part 1}


\begin{document}

\maketitle
\chapter{\MyTitle}
%\begin{flushright}
%        \MyAuthor, \MyDate
%\end{flushright}


\section{Introduction to Security Architecture}
\begin{figure}[h]
\centering
\includegraphics[width=1\linewidth]{res/SecurityLayers}
\caption{Security Layers in a common JEE application}
\label{fig:SecurityLayers}
\end{figure}


\section{Authentication}
Authentication\\
Who are you?\\
Identification\\

\section{Authorization}
Authorization\\
What are you allowed to do?\\
Assignment of Permissions to a Authenticated User\\


\section{Deployment Descriptors}
Describes how the Application should be Deployed.\\
Defines Security Constraints\\
\begin{itemize}
	\item Protected Information
	\item Probably SSL
	\item Specify which user may access them
\end{itemize}
Deployment Descriptors are XML-Files\\
Usually located in /WEB-INF/\\
\begin{itemize}
	\item web.xml
	\item Vendor-specific.xml (E.g. Glassfish: glassfish-web.xml)
\end{itemize}

\subsection*{web.xml}
Protected Resources\\
Security Roles\\
Authentication methods\\

\subsection*{(vendor-specific).xml}
User – Role mapping\\
Group – Role mapping\\
\\
Vendor specific settings\\

\section{Principals}
A Principal is a identity that can be authenticated.\\
E.g. a Unique user name\\

\section{Credential}
A Credential is defined as information that is used to authenticate a Principal.\\
E.g. a Password\\

\section{Groups}
Groups and Principals can be mapped to Roles.\\
Groups are defined in vendor-specific.xml\\

\section{Roles}
Permissions are granted to Roles.\\
Roles are defined in the web.xml file\\

\section{Realms}
aka Security policy domain\\
Provides information about principals, their Groups and their credentials\\
May be a Database, File structure, connection…\\\\
In other words:\\
It contains user information\\
E.g. Username, Password \& Permissions\\

\newpage
\section{Implementation sample}
\href{https://github.com/aayvazyan-tgm/JavaEESecurityExample}{https://github.com/aayvazyan-tgm/JavaEESecurityExample}\\
\begin{figure}[h!]
\centering
\includegraphics[width=1\linewidth]{res/Unauthorized}
\caption{The user tries to access a resource without authentication}
\label{fig:Unauthorized}
\end{figure}

\begin{figure}[h!]
	\centering
	\includegraphics[width=1\linewidth]{res/Authorized}
	\caption{The user sends authentication data with his request}
	\label{fig:Authorized}
\end{figure}

\section{Frameworks}
\subsection{Shiro}
Offers: Authentication, Authorization, Cryptography\\
Simple to use\\\\
Advantages/Disadvantages\\
Implementation Sample

\subsection{Spring}
Offers: Authentication, Authorization, Cryptography\\
Very structured\\\\
Advantages/Disadvantages\\

\subsection{JAAS - Java Authentication and Authorization Service}
Offers: Authentication, Authorization, Cryptography\\
Included in Java SE since Java 1.4 (javax.security.auth)\\\\
Advantages/Disadvantages\\

\section{Output escaping}
Escape user input to prevent injections.\\
\\
Escape the output to add a extra layer of security.\\
Use a Framework to do so!\\

\section{Whats to come in Part 2 (Adrian)}
\begin{itemize}
	\item Working with Digital Certificates\\
	\item Securing Application Clients\\
	\item Security with Enterprise Beans\\
	\item Further Framework Information\\
\end{itemize}


\newpage
\section*{Sources}
JavaOne 2014: The Anatomy of a Secure Web Application Using Java, \\
Shawn McKinney \& John Field, September 29, 2014\\
San Francisco\\\\
Java Security: Sicherheitslücken identifizieren und vermeiden, \\
Marc Schönefeld, 1. edition 2011\\
Publisher: Hüthig Jehle Rehm GmbH, Heidelberg. \\
ISBN/ISSN 978-3-8266-9105-8\\\\
Enterprise Java Security: Building Secure J2EE Applications,\\
Marco Pistoia, Nataraj Nagaratnam, Larry Koved, Anthony Nadalin,\\
1. edition 2004 \\
Publisher: Addison-Wesley Professional.\\
ISBN/ISSN: ISBN 0-321-11889-8\\\\
Official JavaEE Documentation, Oracle,\\
29.09.2014
http://docs.oracle.com/javaee/7/tutorial/doc/security-intro.htm\\\\
Java EE 6,\\
Dirk Weil, 1. edition 2012\\
Publisher: entwickler.press\\
ISBN 978-3-86802-077-9\\\\
Java EE 6 Cookbook for Securing, Tuning, and Extending Enterprise Applications,\\
Mick Knutson,\\
1. edition June 2012 \\
Publisher: Addison-Wesley Professional.\\
ISBN/ISSN: ISBN 9781849683166\\\\
\end{document}
